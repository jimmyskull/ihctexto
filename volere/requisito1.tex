
%%%%%%%%%%%%%%%%%%%%%%%%%%%%%%%%%%%%%%%%%%%%%%%%%%%%%%%%%%%%%%%%%%%%%%%%%%
%%%%%%%%%%%%%%%%%%%%%%%%%%%%% TEMPLATE VOLERE %%%%%%%%%%%%%%%%%%%%%%%%%%%%
\par \noindent
\begin{table}[h]
    \begin{tabularx}{\textwidth}{
        p{\dimexpr.18\linewidth-2\tabcolsep-1.3333\arrayrulewidth}
        p{\dimexpr.1\linewidth-2\tabcolsep-1.3333\arrayrulewidth}
        p{\dimexpr.24\linewidth-2\tabcolsep-1.3333\arrayrulewidth}
        p{\dimexpr.14\linewidth-2\tabcolsep-1.3333\arrayrulewidth}
        p{\dimexpr.255\linewidth-2\tabcolsep-1.3333\arrayrulewidth}
        p{\dimexpr.1\linewidth-2\tabcolsep-1.3333\arrayrulewidth}
    }
    \toprule
        \textbf{Requisito \#:}          & 1 
        & \textbf{Tipo de requisito:}   & 1
        & \textbf{Evento/PUC/BUC:}      & 1, 2 \\
    \midrule
    \end{tabularx}
    \vspace{5pt}
    \begin{tabularx}{\textwidth}{p{.12\textwidth}  p{.83\textwidth}}
        \textbf{Descrição:} &
        Integração com as câmeras internas do \emph{smartphone}.
    \end{tabularx}
    
    \vspace{5pt}
    \begin{tabularx}{\textwidth}{p{.12\textwidth}  p{.83\textwidth}}
        \textbf{Razão:} &
        Utilizar as imagens geradas pela câmera como entrada para a 
        busca pelo produto no \emph{software}.
    \end{tabularx}
    
    \vspace{5pt}
    \begin{tabularx}{\textwidth}{p{.12\textwidth} p{.83\textwidth}}
        \textbf{Origem:} &
        Lucas.
    \end{tabularx}
    
    \vspace{5pt}
    \begin{tabularx}{\textwidth}{p{.23\textwidth} p{.72\textwidth}}
        \textbf{Critério de ajuste:} &
        A integração decorre no sentido de que o \emph{software} deve 
        ser capaz de captar as imagens da câmera do usuário quando 
        eles tiram uma foto, independente se é utilizando o sistema 
        para tirar foto ou o próprio \emph{software} de fotos do 
        \emph{smartphone}. 
    \end{tabularx}
    
    \vspace{5pt}
    \begin{tabularx}{\textwidth}{p{.26\textwidth}p{.248\textwidth}
                                 p{.28\textwidth}p{.110\textwidth}}
        \textbf{Satisfação do cliente:} & -- & 
        \textbf{Insatisfação do cliente:} & --
    \end{tabularx}
    
    \vspace{5pt}
    \begin{tabularx}{\textwidth}{p{.17\textwidth}p{.5\textwidth}
                                 p{.12\textwidth}p{.101\textwidth}}
        \textbf{Dependências:} & Depende do acesso ao módulo da 
            câmera via API do \emph{android}.
        & \textbf{Conflitos:} & -- \\
    \end{tabularx}
    
    \vspace{5pt}
    \begin{tabularx}{\textwidth}{p{.235\textwidth} p{.715\textwidth}}
        \textbf{Materiais de apoio:} &
        Documentação da porção do módulo da câmera interna, encontrado
        na documentação para desenvolvedor do \emph{android}.
    \end{tabularx}
    
    \vspace{5pt}
    \begin{tabularx}{\textwidth}{p{.1\textwidth} p{.85\textwidth}}
        \textbf{História:} &
        Criado em 12 de Junho de 2012. \\
    \end{tabularx}
    
    \begin{tabularx}{\textwidth}{c}
        \bottomrule
    \end{tabularx}
    
    \caption{ \label{Volere1} Requisito no modelo de Volere. }
\end{table}
%%%%%%%%%%%%%%%%%%%%%%%%%%%%%%%%%%%%%%%%%%%%%%%%%%%%%%%%%%%%%%%%%%%%%%%%%%

