
%{\color{red} Texto introdutório para as personas.}

Uma persona, conceito introduzido por \citeonline{cooper2004inmates}, define o 
arquétipo de um usuário de um sistema, um exemplo de um tipo potencial
de pessoa que pode interagir o sistema.  A ideia de uma persona é que
para projetar um \emph{software} efetivo, é necessário que este o seja
projetado para um tipo específico de pessoa.  As personas devem 
representar pessoas fictícias que são baseadas no conhecimento de
pessoas reais.

Nesta seção são descritas três personas.  Foi explorado o foco em pessoas
as quais irão utilizar o aplicativo em situações cotidianas, onde o usuário
pode possivelmente utilizar o aplicativo em paralelo com outras atividades
não relacionadas ao aplicativo.

%\subsection{Persona 1}  \label{sec:persona1}

%%%%%%%%%%%%%%%%%%%%%%%%%%%%% PERSONA %%%%%%%%%%%%%%%%%%%%%%%%%%%%%%%%%%%
\par \noindent
\begin{table}[!ht] 
    \begin{tabularx}{\textwidth}{c}
        \toprule
    \end{tabularx}
    
    \begin{tabularx}{\textwidth}{p{.17\textwidth}p{.5\textwidth}
                                 p{.12\textwidth}p{.101\textwidth}}
        \textbf{Nome:} & Eduardo H. de Mello
        & \textbf{Idade:} & 24 anos \\
    \midrule
    \end{tabularx}
    
    \hspace*{1cm}Eduardo é um rapaz recém formado em Arquitetura e Urbanismo que está de 
mudança para Curitiba, onde foi aprovado no programa de Mestrado da UFPR.
Eduardo necessita comparar preços de livros que possuem poucos exemplares na biblioteca para auxilio durante o mestrado..

\hspace*{1cm}Como planeja realizar compras na cidade de Curitiba -- PR, 
torna-se inexequível a procura do melhor preço em todas as lojas factíveis. 
Não obstante, deseja obter os produtos com os melhores preços, fato que
implica na necessidade de analisar valores em lojas locais e virtuais.
Discretamente, vai à biblioteca s e tira fotos de um
livro de  seu interesse.
Devido as condições da biblioteca,
localiza o código QR do livro.
Ao tirar uma foto do código QR, se estiver conectado à \textit{internet},
em apenas três segundos já deve começar a obter respostas da consulta
para o produto.

\hspace*{1cm}Com os resultados recém consultados, conseguirá analisar os melhores
preços e ler avaliações dos produtos em que teve interesse.  Desta forma, 
poderá saber o local do melhor preço.

\hspace*{1cm}A expectativa é que os preços dos produtos estejam certos, pois isto pode 
fazer com que alguns vendedores das lojas da sua localidade melhorem os preços
em geral, evitando preços muito aquém da média. No decorrer da avalição, como não possui bagagem informativa
sobre as características de eletrodomésticos, além do fato de 
precisar de alguns, a qualificação dos produtos avaliados por outros usuários
como ele, o influenciam na compra de um ou outro determinado produto dentre os
ofertados.
    
    \begin{tabularx}{\textwidth}{c}
        \bottomrule
    \end{tabularx}
    
    \caption{ \label{Persona1} Persona 1}
\end{table}
%%%%%%%%%%%%%%%%%%%%%%%%%%%%% PERSONA %%%%%%%%%%%%%%%%%%%%%%%%%%%%%%%%%%%

%\FloatBarrier
%\subsection{Persona 2}  \label{sec:persona2}
%%%%%%%%%%%%%%%%%%%%%%%%%%%%% PERSONA %%%%%%%%%%%%%%%%%%%%%%%%%%%%%%%%%%%
\par \noindent
\begin{table}[ht]
    \begin{tabularx}{\textwidth}{c}
        \toprule
    \end{tabularx}
    
    \begin{tabularx}{\textwidth}{p{.17\textwidth}p{.5\textwidth}
                                 p{.12\textwidth}p{.101\textwidth}}
        \textbf{Nome:} & Clotilde Van Halen 
        & \textbf{Idade:} & 43 anos \\
    \midrule
    \end{tabularx}
    
    \hspace*{1cm} Clotilde é uma mulher bem sucedida, chefe de família, 
    presidente do grupo de leitura do bairro e gerente geral de uma 
    agência bancária.  Precisou absorver o conhecimento de operação
    de sistemas computacionais desde que começou a trabalhar banco.
    Clotilde, formada em Ciências Econômicas, também passou a ter interesse 
    por tecnologia para pode acompanhar as constantes mudanças que ocorrem 
    no mercado.

    \hspace*{1cm} Ela deseja comprar um \textit DVD do Justin Bieber para sua filha,
    Kimberly, como presente de aniversário de 17 anos.  Clotilde está
    feliz, pois sua filha acabou de ser aprovada no vestibular de Direito na UNIFACIL
    e deseja recompensar sua filha com um presente surpresa.  
    
    \hspace*{1cm} Como a janela de almoço é de apenas uma hora, em horário comercial, 
    deseja localizar rapidamente onde deve comprar um DVD do Justin igual o da filha de  sua 
    colega, Priscila.  Priscila recomenda a Clotilde, a utilização do aplicativo
    Doutor Pecúlio, para que possa descobrir facilmente os locais de venda
    do DVD.   Ao instalar o aplicativo, Clotilde tira uma foto
    do DVD  da filha de  Priscila.
    Desta forma, poderá decidir se vale a pena comprar em uma loja próxima
    ao banco, ou se vale a pena efetuar o pedido via
    \textit{internet}.

\hspace*{1cm}Ao final, aproveita para qualificar o produto escolhido conforme avaliação feita após a consulta 
feita pelo aplicativo. 
    
    \begin{tabularx}{\textwidth}{c}
        \bottomrule
    \end{tabularx}
    
    \caption{ \label{Persona2} Persona 2}
\end{table}
%%%%%%%%%%%%%%%%%%%%%%%%%%%%% PERSONA %%%%%%%%%%%%%%%%%%%%%%%%%%%%%%%%%%%

\FloatBarrier
%\subsection{Persona 3}  \label{sec:persona3}
%%%%%%%%%%%%%%%%%%%%%%%%%%%%% PERSONA %%%%%%%%%%%%%%%%%%%%%%%%%%%%%%%%%%%
\par \noindent
\begin{table}[ht]
    \begin{tabularx}{\textwidth}{c}
        \toprule
    \end{tabularx}
    
    \begin{tabularx}{\textwidth}{p{.17\textwidth}p{.5\textwidth}
                                 p{.12\textwidth}p{.101\textwidth}}
        \textbf{Nome:} & Sirlei da Silva
        & \textbf{Idade:} & 57 anos \\
    \midrule
    \end{tabularx}
    
    \hspace*{1cm} Sirlei é um aposentado que recentemente criou o hábito de ir
    à locadora ver as novidades do cinema. Para não se perder na  megalópoles com seus 
    colegas, utiliza um \textit{smartphone} como GPS (do inglês,
    \textit{Global Positioning System}), porém, o utiliza com óbice,
    não obstante, sempre concretiza as tarefas.

    \hspace*{1cm} Deseja tirar a foto das capas dos filmes para saber qual 
    a média dos preços dos mesmo e saber aproximadamente quando deve  
    recomendar a compra de um filme para alguem ou não. Sirlei incomoda-se de ter que consultar o dono da locadora
    constantemente, pois é uma tarefa extenuante. 
    
    \hspace*{1cm} Deseja que, no fim das contas, possa acessar o registro 
    de todas as consultas feitas enquanto analisava os carros no pátio, 
    no decorrer de cada foto tirada.

    \hspace*{1cm} o caso deste tipo de produto, os anúncios costumam ser 
    temporais e  sempre existem qualificações para os filmes.

    \begin{tabularx}{\textwidth}{c}
        \bottomrule
    \end{tabularx}
    
    \caption{ \label{Persona3} Persona 3}
\end{table}
%%%%%%%%%%%%%%%%%%%%%%%%%%%%% PERSONA %%%%%%%%%%%%%%%%%%%%%%%%%%%%%%%%%%%
