
%{\color{red}
%• Um plano de avaliação da interface, deverá ser feita quanto a usabilidade, 
%comunicabilidade e acessibilidade.}

As tecnologias de informação e comunicação oferecem as maneiras
mais eficientes de manipular informações em suas mais variadas formas.
\citeonline{Barbosa} focam em sistemas computacionais interativos,
que fazem cada vez mais parte do cotidiano de todos os usuários.
Em termos de \textit{design} de sistemas móveis, deve-se dar ênfase 
no usuário.  Segundo \citeonline{love2005understanding}, primeiramente,
o \textit{designer} do sistema deve compreender, do início, o que os
usuários vão querer utilizar em um dispositivo móvel, ou seja, quais 
tarefas ele irá realizar enquanto estiver utilizando o sistema.  Em
segundo lugar, quais características do usuário podem afetar
significativamente a usabilidade no sistema, como idade e deficiências
físicas, como por exemplo, cegueira.   Em terceiro lugar, uma vez que o \textit{designer}
tenha levado em consideração as necessidades do usuário, no próximo estágio
deve ser desenvolvido um sistema que adeque-se aos
requisitos identificados na primeira fase.  Então o \textit{designer}
deve testar o sistema para avaliar se o sistema atende às necessidades e
se a utilização é satisfatória.  Com base no \textit{feedback} recebido
durante a avaliação, o \textit{designer} deve produzir uma versão a
atualizada.

Outra característica deve ser levada em conta por ter um grande impacto,
é o contexto de uso, ou seja, em quais ambientes os usuários estarão
ao utilizar o sistema, que afetam na usabilidade, como habilidade, 
eficiência e satisfação ao utilizar.

\subsection{Metas de usabilidade}

%{\color{red}
%• Uma lista com as metas de usabilidade e experiência do usuário a serem atingidas pela 
%aplicação. }

% Um agradecimento ao autor original
% http://www.csl.mtu.edu/cs5760/www/Lectures/CurrentLectures/Usability%20and%20Suite%20Consistency.htm
% :P

\citeonline{rogers2011interaction} sugerem seis metas de usabilidade:
eficácia, eficiência, utilidade, facilidade de aprendizado, capacidade
de memorização e segurança.
Eficácia e utilidade referem-se à funcionalidade do aplicativo.
A eficácia é uma medida geral de como o sistema executa, já a utilidade, é
uma medida das funcionalidades implementadas corretamente e a abrangência
delas.  A eficiência refere-se ao tempo necessário para
utilizar a \textit{interface} e a probabilidade de cometer erros iteragindo
com o sistema.  \textit{Designers} são encorajados a criar 
\textit{interfaces} familiares e naturais, com foco na facilidade de
aprendizado e memorização, ou seja, que possam ser aprendidas
sem a necessidade da leitura de um manual.   No contexto de desenvolvimento
móvel, pouco se discute sobre segurança como uma meta principal de 
usabilidade, pois está mais associada com a garantia de que o usuário
não possa por o sistema em condições periosas do que a segurança como 
consistência de dados.

Como metas pragmáticas de usabilidade para o aplicativo móvel Doutor Pecúlio,
planejamos dar foco em:

\begin{description}
    \item[Aprendizado:] o usuário geralmente não tem tempo e dificilmente
        irá procurar o manual de um aplicativo, o que implica em ter 
        uma \textit{interface} com curva de aprendizagem tênue;
    \item[eficiência:] o usuário não planeja gastar muito tempo
        realizando uma tarefa complexa, portanto, deve ser mantido o
        foco em funcionalidades simples e que possam ser efetuadas
        rapidamente; e
    \item[ergonomia:] as telas dos dispositivos são pequenas, para que o 
        usuário consiga manipular utilizando apenas uma mão, por isso deve
        ser fácil de navegar.
\end{description}

\subsection{Experiência do usuário}

\citeonline{rogers2011interaction} identificam que os objetivos em uma
\textit{interface} gráfica são focados na usabilidade e funcionalidade.
Eles sugerem como metas para experiência a satisfação, agradabilidade,
diversão, entretenimento, prestativo, motivador, estético, motiva
criatividade, recompensador e emocionalmente gratificante.

Para o aplicativo Doutor Pecúlio, selecionamos cinco metas:

\begin{description}
    \item[Prestativo:] o aplicativo deve ter como objetivo ser útil ao 
        usuário em tarefas corriqueiras;
    \item[motivador:] o usuário deve se interessar para utilizar;
    \item[recompensador:] a recompensa irá certificar que o usuário 
        continuará utilizando, após uma interação bem sucedida ele saberá
        que a qualquer momento poderá interagir novamente; e
    \item[agradabilidade:] a taxa de erros de consulta deve ser muito 
        baixa para evitar que o usuário se fruste.
\end{description}
