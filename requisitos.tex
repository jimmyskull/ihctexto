
A partir dos cenários e casos de uso foram identificados quatro
requisitos funcionais e três requisitos não funcionais.
Os requisitos funcionais são:

\begin{itemize}
\item Integração com as câmeras internas do \emph{smartphone};
\item integração com as APIs de compartilhamento de multimídia;
\item integração com a API GeoIP; e
\item integração com os serviços de lojas virtuais (Amazon.com, Livraria
    Saraiva, Livrarias Curitiba, Livraria Chain, Submarino.com.br).
\end{itemize}

Os requisitos não funcionais são:
\begin{itemize}
\item Acesso à \emph{internet};
\item visualização do histórico de buscas; e
\item acesso ao catálogo de lojas próximas, com base nos resultados
    da API GeoIP.
\end{itemize}

\subsection{Requisitos no modelo de Volere}

%{\color{red}
%• Um conjunto de especificação dos requisitos usando a template do sistema Volere. }

Volere, criado por \citeonline{Robertson:1999:MRP:312381}, é uma coleção de recursos
para levantamento de requisitos, voltada para ajudar a descobrir, compreender,
escrever e comunicar requisitos \cite{VolereOverview}.
Através de modelos flexíveis, o objetivo é melhorar a consistência da estimativa de
tempo, de controle de riscos, monitoramento e reutilização de dados 
analisados.  Foram preenchidos sete requisitos utilizando o modelo de Volere 
disponível \emph{online}\footnote{Disponível em:
<http://www.volere.co.uk/languages.htm>. Acesso em: 11 jun. 2013.}, 
os quais podem ser vistos nas 
tabelas \ref{Volere1}, \ref{Volere2}, \ref{Volere3}, \ref{Volere4},
\ref{Volere5}, \ref{Volere6} e \ref{Volere7}.  Cada requisito contém vários
campos.  Na primeira linha o número do requisito, o tipo de requisito, se este 
é um (1) requisito funcional ou (2) requisito não funcional, e a lista de
casos de uso relacionados.   As três linhas seguintes reportam a
descrição, o nome do requisito, a razão, o motivo de este ser alavancado 
como um requisito e a origem, o nome do autor responsável por propor
o requisito.  Na quinta linha, o critério de ajuste é uma medida que tem
por objetivo propor uma solução para verificar se esta é apropriada às exigências
do requisito.  A satisfação é a quantificação em uma escala de satisfação do 
cliente em relação ao requisito.  Neste trabalho não foram utilizadas escalas
de satisfação, porém os campos foram mantidos para cumprir o modelo.  A sétima
linha lista as dependências para o requisito e a lista de requisitos que são
conflitantes.  As duas últimas linhas reportam os materiais de apoio factíveis
e o histórico de alterações do requisito.


%%%%%%%%%%%%%%%%%%%%%%%%%%%%%%%%%%%%%%%%%%%%%%%%%%%%%%%%%%%%%%%%%%%%%%%%%%
%%%%%%%%%%%%%%%%%%%%%%%%%%%%% TEMPLATE VOLERE %%%%%%%%%%%%%%%%%%%%%%%%%%%%
\par \noindent
\begin{table}[h]
    \begin{tabularx}{\textwidth}{
        p{\dimexpr.18\linewidth-2\tabcolsep-1.3333\arrayrulewidth}
        p{\dimexpr.1\linewidth-2\tabcolsep-1.3333\arrayrulewidth}
        p{\dimexpr.24\linewidth-2\tabcolsep-1.3333\arrayrulewidth}
        p{\dimexpr.14\linewidth-2\tabcolsep-1.3333\arrayrulewidth}
        p{\dimexpr.255\linewidth-2\tabcolsep-1.3333\arrayrulewidth}
        p{\dimexpr.1\linewidth-2\tabcolsep-1.3333\arrayrulewidth}
    }
    \toprule
        \textbf{Requisito \#:}          & 1 
        & \textbf{Tipo de requisito:}   & 1
        & \textbf{Evento/PUC/BUC:}      & 1, 2 \\
    \midrule
    \end{tabularx}
    \vspace{5pt}
    \begin{tabularx}{\textwidth}{p{.12\textwidth}  p{.83\textwidth}}
        \textbf{Descrição:} &
        Integração com as câmeras internas do \emph{smartphone}.
    \end{tabularx}
    
    \vspace{5pt}
    \begin{tabularx}{\textwidth}{p{.12\textwidth}  p{.83\textwidth}}
        \textbf{Razão:} &
        Utilizar as imagens geradas pela câmera como entrada para a 
        busca pelo produto no \emph{software}.
    \end{tabularx}
    
    \vspace{5pt}
    \begin{tabularx}{\textwidth}{p{.12\textwidth} p{.83\textwidth}}
        \textbf{Origem:} &
        Lucas.
    \end{tabularx}
    
    \vspace{5pt}
    \begin{tabularx}{\textwidth}{p{.23\textwidth} p{.72\textwidth}}
        \textbf{Critério de ajuste:} &
        A integração decorre no sentido de que o \emph{software} deve 
        ser capaz de captar as imagens da câmera do usuário quando 
        eles tiram uma foto, independente se é utilizando o sistema 
        para tirar foto ou o próprio \emph{software} de fotos do 
        \emph{smartphone}. 
    \end{tabularx}
    
    \vspace{5pt}
    \begin{tabularx}{\textwidth}{p{.26\textwidth}p{.248\textwidth}
                                 p{.28\textwidth}p{.110\textwidth}}
        \textbf{Satisfação do cliente:} & -- & 
        \textbf{Insatisfação do cliente:} & --
    \end{tabularx}
    
    \vspace{5pt}
    \begin{tabularx}{\textwidth}{p{.17\textwidth}p{.5\textwidth}
                                 p{.12\textwidth}p{.101\textwidth}}
        \textbf{Dependências:} & Depende do acesso ao módulo da 
            câmera via API do \emph{android}.
        & \textbf{Conflitos:} & -- \\
    \end{tabularx}
    
    \vspace{5pt}
    \begin{tabularx}{\textwidth}{p{.235\textwidth} p{.715\textwidth}}
        \textbf{Materiais de apoio:} &
        Documentação da porção do módulo da câmera interna, encontrado
        na documentação para desenvolvedor do \emph{android}.
    \end{tabularx}
    
    \vspace{5pt}
    \begin{tabularx}{\textwidth}{p{.1\textwidth} p{.85\textwidth}}
        \textbf{História:} &
        Criado em 12 de Junho de 2012. \\
    \end{tabularx}
    
    \begin{tabularx}{\textwidth}{c}
        \bottomrule
    \end{tabularx}
    
    \caption{ \label{Volere1} Requisito no modelo de Volere. }
\end{table}
%%%%%%%%%%%%%%%%%%%%%%%%%%%%%%%%%%%%%%%%%%%%%%%%%%%%%%%%%%%%%%%%%%%%%%%%%%



%%%%%%%%%%%%%%%%%%%%%%%%%%%%%%%%%%%%%%%%%%%%%%%%%%%%%%%%%%%%%%%%%%%%%%%%%%
%%%%%%%%%%%%%%%%%%%%%%%%%%%%% TEMPLATE VOLERE %%%%%%%%%%%%%%%%%%%%%%%%%%%%
\par \noindent
\begin{table}[h]
    \begin{tabularx}{\textwidth}{
        p{\dimexpr.18\linewidth-2\tabcolsep-1.3333\arrayrulewidth}
        p{\dimexpr.1\linewidth-2\tabcolsep-1.3333\arrayrulewidth}
        p{\dimexpr.24\linewidth-2\tabcolsep-1.3333\arrayrulewidth}
        p{\dimexpr.14\linewidth-2\tabcolsep-1.3333\arrayrulewidth}
        p{\dimexpr.255\linewidth-2\tabcolsep-1.3333\arrayrulewidth}
        p{\dimexpr.1\linewidth-2\tabcolsep-1.3333\arrayrulewidth}
    }
    \toprule
        \textbf{Requisito \#:}          & 2
        & \textbf{Tipo de requisito:}   & 1
        & \textbf{Evento/PUC/BUC:}      & 1, 2 \\
    \midrule
    \end{tabularx}
    \vspace{5pt}
    \begin{tabularx}{\textwidth}{p{.12\textwidth}  p{.83\textwidth}}
        \textbf{Descrição:} &
        Integração com as APIs de compartilhamento de multimídia.
    \end{tabularx}
    
    \vspace{5pt}
    \begin{tabularx}{\textwidth}{p{.12\textwidth}  p{.83\textwidth}}
        \textbf{Razão:} &
        Para poder usar o aplicativo após fotos tirada pelo aplicativo
        externo de fotos do \emph{smartphone}.
    \end{tabularx}
    
    \vspace{5pt}
    \begin{tabularx}{\textwidth}{p{.12\textwidth} p{.83\textwidth}}
        \textbf{Origem:} &
        Paulo.
    \end{tabularx}
    
    \vspace{5pt}
    \begin{tabularx}{\textwidth}{p{.23\textwidth} p{.72\textwidth}}
        \textbf{Critério de ajuste:} &
        Deve-se desenvolver, de acordo com o as resoluções provenientes 
        da API do sistema operacional a inserção da opção 
        Doutor Pecúlio no menu de aplicativos de compartilhamento,
        o qual direcione a foto ao Doutor Pecúlio diretamente para a
        fase de busca do produto.
    \end{tabularx}
    
    \vspace{5pt}
    \begin{tabularx}{\textwidth}{p{.26\textwidth}p{.248\textwidth}
                                 p{.28\textwidth}p{.110\textwidth}}
        \textbf{Satisfação do cliente:} & -- & 
        \textbf{Insatisfação do cliente:} & --
    \end{tabularx}
    
    \vspace{5pt}
    \begin{tabularx}{\textwidth}{p{.17\textwidth}p{.5\textwidth}
                                 p{.12\textwidth}p{.101\textwidth}}
        \textbf{Dependências:} & Depende de como a API do \emph{android}
        é implementada.
        & \textbf{Conflitos:} & -- \\
    \end{tabularx}
    
    \vspace{5pt}
    \begin{tabularx}{\textwidth}{p{.235\textwidth} p{.715\textwidth}}
        \textbf{Materiais de apoio:} &
       Documentação da API do módulo da lista de opções de 
       compartilhamento do \emph{android}.
    \end{tabularx}
    
    \vspace{5pt}
    \begin{tabularx}{\textwidth}{p{.1\textwidth} p{.85\textwidth}}
        \textbf{História:} &
        Criado em 12 de Junho de 2012. \\
    \end{tabularx}
    
    \begin{tabularx}{\textwidth}{c}
        \bottomrule
    \end{tabularx}
    
    \caption{ \label{Volere2} Requisito no modelo de Volere. }
\end{table}
%%%%%%%%%%%%%%%%%%%%%%%%%%%%%%%%%%%%%%%%%%%%%%%%%%%%%%%%%%%%%%%%%%%%%%%%%%


%%%%%%%%%%%%%%%%%%%%%%%%%%%%%%%%%%%%%%%%%%%%%%%%%%%%%%%%%%%%%%%%%%%%%%%%%%
%%%%%%%%%%%%%%%%%%%%%%%%%%%%% TEMPLATE VOLERE %%%%%%%%%%%%%%%%%%%%%%%%%%%%
\par \noindent
\begin{table}[h]
    \begin{tabularx}{\textwidth}{
        p{\dimexpr.18\linewidth-2\tabcolsep-1.3333\arrayrulewidth}
        p{\dimexpr.1\linewidth-2\tabcolsep-1.3333\arrayrulewidth}
        p{\dimexpr.24\linewidth-2\tabcolsep-1.3333\arrayrulewidth}
        p{\dimexpr.14\linewidth-2\tabcolsep-1.3333\arrayrulewidth}
        p{\dimexpr.255\linewidth-2\tabcolsep-1.3333\arrayrulewidth}
        p{\dimexpr.1\linewidth-2\tabcolsep-1.3333\arrayrulewidth}
    }
    \toprule
        \textbf{Requisito \#:}          & 3
        & \textbf{Tipo de requisito:}   & 1
        & \textbf{Evento/PUC/BUC:}      & 1 \\
    \midrule
    \end{tabularx}
    \vspace{5pt}
    \begin{tabularx}{\textwidth}{p{.12\textwidth}  p{.83\textwidth}}
        \textbf{Descrição:} &
        Integração com a API GeoIP.
    \end{tabularx}
    
    \vspace{5pt}
    \begin{tabularx}{\textwidth}{p{.12\textwidth}  p{.83\textwidth}}
        \textbf{Razão:} &
        Obter informação da localização atual do usuário,
        utilizando a API GeoIP.
    \end{tabularx}
    
    \vspace{5pt}
    \begin{tabularx}{\textwidth}{p{.12\textwidth} p{.83\textwidth}}
        \textbf{Origem:} &
        Lucas.
    \end{tabularx}
    
    \vspace{5pt}
    \begin{tabularx}{\textwidth}{p{.23\textwidth} p{.72\textwidth}}
        \textbf{Critério de ajuste:} &
        O aplicativo deve ter acesso a informação do identificador
        IP da conexão atual do \emph{smartphone} para poder,
        através do GeoIP, descobrir a localização atual e então
        utilizar como critério de busca de catálogos de lojas
        próximas.
    \end{tabularx}
    
    \vspace{5pt}
    \begin{tabularx}{\textwidth}{p{.26\textwidth}p{.248\textwidth}
                                 p{.28\textwidth}p{.110\textwidth}}
        \textbf{Satisfação do cliente:} & -- & 
        \textbf{Insatisfação do cliente:} & --
    \end{tabularx}
    
    \vspace{5pt}
    \begin{tabularx}{\textwidth}{p{.17\textwidth}p{.5\textwidth}
                                 p{.12\textwidth}p{.101\textwidth}}
        \textbf{Dependências:} & Conseguir adquirir o endereço de IP da
        conexão atual.
        & \textbf{Conflitos:} & -- \\
    \end{tabularx}
    
    \vspace{5pt}
    \begin{tabularx}{\textwidth}{p{.235\textwidth} p{.715\textwidth}}
        \textbf{Materiais de apoio:} &
       Documentação da API do GeoIP.
    \end{tabularx}
    
    \vspace{5pt}
    \begin{tabularx}{\textwidth}{p{.1\textwidth} p{.85\textwidth}}
        \textbf{História:} &
        Criado em 12 de Junho de 2012. \\
    \end{tabularx}
    
    \begin{tabularx}{\textwidth}{c}
        \bottomrule
    \end{tabularx}
    
    \caption{ \label{Volere3} Requisito no modelo de Volere. }
\end{table}
%%%%%%%%%%%%%%%%%%%%%%%%%%%%%%%%%%%%%%%%%%%%%%%%%%%%%%%%%%%%%%%%%%%%%%%%%%


%%%%%%%%%%%%%%%%%%%%%%%%%%%%%%%%%%%%%%%%%%%%%%%%%%%%%%%%%%%%%%%%%%%%%%%%%%
%%%%%%%%%%%%%%%%%%%%%%%%%%%%% TEMPLATE VOLERE %%%%%%%%%%%%%%%%%%%%%%%%%%%%
\par \noindent
\begin{table}[h]
    \begin{tabularx}{\textwidth}{
        p{\dimexpr.18\linewidth-2\tabcolsep-1.3333\arrayrulewidth}
        p{\dimexpr.1\linewidth-2\tabcolsep-1.3333\arrayrulewidth}
        p{\dimexpr.24\linewidth-2\tabcolsep-1.3333\arrayrulewidth}
        p{\dimexpr.14\linewidth-2\tabcolsep-1.3333\arrayrulewidth}
        p{\dimexpr.255\linewidth-2\tabcolsep-1.3333\arrayrulewidth}
        p{\dimexpr.1\linewidth-2\tabcolsep-1.3333\arrayrulewidth}
    }
    \toprule
        \textbf{Requisito \#:}          & 4
        & \textbf{Tipo de requisito:}   & 1
        & \textbf{Evento/PUC/BUC:}      & 1 \\
    \midrule
    \end{tabularx}
    \vspace{5pt}
    \begin{tabularx}{\textwidth}{p{.12\textwidth}  p{.83\textwidth}}
        \textbf{Descrição:} &
        Integração com os serviços de lojas virtuais (Amazon.com, 
        Livraria Saraiva, Livrarias Curitiba, Livraria Chain, 
        Submarino.com.br).
    \end{tabularx}
    
    \vspace{5pt}
    \begin{tabularx}{\textwidth}{p{.12\textwidth}  p{.83\textwidth}}
        \textbf{Razão:} &
        Poder consultar preços dos livros, CD's, DVD's, BD's, etc.
    \end{tabularx}
    
    \vspace{5pt}
    \begin{tabularx}{\textwidth}{p{.12\textwidth} p{.83\textwidth}}
        \textbf{Origem:} &
        Carlos.
    \end{tabularx}
    
    \vspace{5pt}
    \begin{tabularx}{\textwidth}{p{.23\textwidth} p{.72\textwidth}}
        \textbf{Critério de ajuste:} &
        Todas as livrarias que tenham disponibilizados seus 
        catálogos na \emph{internet}.
    \end{tabularx}
    
    \vspace{5pt}
    \begin{tabularx}{\textwidth}{p{.26\textwidth}p{.248\textwidth}
                                 p{.28\textwidth}p{.110\textwidth}}
        \textbf{Satisfação do cliente:} & -- & 
        \textbf{Insatisfação do cliente:} & --
    \end{tabularx}
    
    \vspace{5pt}
    \begin{tabularx}{\textwidth}{p{.17\textwidth}p{.5\textwidth}
                                 p{.12\textwidth}p{.101\textwidth}}
        \textbf{Dependências:} & Disponibilidade dos dados das 
        livrarias.  Necessário o título, desejável o preço.
        & \textbf{Conflitos:} & -- \\
    \end{tabularx}
    
    \vspace{5pt}
    \begin{tabularx}{\textwidth}{p{.235\textwidth} p{.715\textwidth}}
        \textbf{Materiais de apoio:} &
       Fornecimento da documentação necessária para integração, 
       de preferência uma API global entre várias lojas, como a
       API do eBay.
    \end{tabularx}
    
    \vspace{5pt}
    \begin{tabularx}{\textwidth}{p{.1\textwidth} p{.85\textwidth}}
        \textbf{História:} &
        Criado em 12 de Junho de 2012. \\
    \end{tabularx}
    
    \begin{tabularx}{\textwidth}{c}
        \bottomrule
    \end{tabularx}
    
    \caption{ \label{Volere4} Requisito no modelo de Volere. }
\end{table}
%%%%%%%%%%%%%%%%%%%%%%%%%%%%%%%%%%%%%%%%%%%%%%%%%%%%%%%%%%%%%%%%%%%%%%%%%%


%%%%%%%%%%%%%%%%%%%%%%%%%%%%%%%%%%%%%%%%%%%%%%%%%%%%%%%%%%%%%%%%%%%%%%%%%%
%%%%%%%%%%%%%%%%%%%%%%%%%%%%% TEMPLATE VOLERE %%%%%%%%%%%%%%%%%%%%%%%%%%%%
\par \noindent
\begin{table}[h]
    \begin{tabularx}{\textwidth}{
        p{\dimexpr.18\linewidth-2\tabcolsep-1.3333\arrayrulewidth}
        p{\dimexpr.1\linewidth-2\tabcolsep-1.3333\arrayrulewidth}
        p{\dimexpr.24\linewidth-2\tabcolsep-1.3333\arrayrulewidth}
        p{\dimexpr.14\linewidth-2\tabcolsep-1.3333\arrayrulewidth}
        p{\dimexpr.255\linewidth-2\tabcolsep-1.3333\arrayrulewidth}
        p{\dimexpr.1\linewidth-2\tabcolsep-1.3333\arrayrulewidth}
    }
    \toprule
        \textbf{Requisito \#:}          & 5
        & \textbf{Tipo de requisito:}   & 2
        & \textbf{Evento/PUC/BUC:}      & 1, 2, 3 \\
    \midrule
    \end{tabularx}
    \vspace{5pt}
    \begin{tabularx}{\textwidth}{p{.12\textwidth}  p{.83\textwidth}}
        \textbf{Descrição:} &
        Acesso à \emph{internet}. 
    \end{tabularx}
    
    \vspace{5pt}
    \begin{tabularx}{\textwidth}{p{.12\textwidth}  p{.83\textwidth}}
        \textbf{Razão:} &
        Consulta de preços em lojas virtuais.
    \end{tabularx}
    
    \vspace{5pt}
    \begin{tabularx}{\textwidth}{p{.12\textwidth} p{.83\textwidth}}
        \textbf{Origem:} &
        Carlos.
    \end{tabularx}
    
    \vspace{5pt}
    \begin{tabularx}{\textwidth}{p{.23\textwidth} p{.72\textwidth}}
        \textbf{Critério de ajuste:} &
        O \emph{smartphone} precisa de mecanismos para conectar à
        um meio de acesso à \emph{internet}, como redes ad-hoc,
        WiFi ou 3G.
    \end{tabularx}
    
    \vspace{5pt}
    \begin{tabularx}{\textwidth}{p{.26\textwidth}p{.248\textwidth}
                                 p{.28\textwidth}p{.110\textwidth}}
        \textbf{Satisfação do cliente:} & --& 
        \textbf{Insatisfação do cliente:} & --
    \end{tabularx}
    
    \vspace{5pt}
    \begin{tabularx}{\textwidth}{p{.17\textwidth}p{.5\textwidth}
                                 p{.12\textwidth}p{.101\textwidth}}
        \textbf{Dependências:} & --
        & \textbf{Conflitos:} & -- \\
    \end{tabularx}
    
    \vspace{5pt}
    \begin{tabularx}{\textwidth}{p{.235\textwidth} p{.715\textwidth}}
        \textbf{Materiais de apoio:} &
       --
    \end{tabularx}
    
    \vspace{5pt}
    \begin{tabularx}{\textwidth}{p{.1\textwidth} p{.85\textwidth}}
        \textbf{História:} &
        Criado em 12 de Junho de 2012. \\
    \end{tabularx}
    
    \begin{tabularx}{\textwidth}{c}
        \bottomrule
    \end{tabularx}
    
    \caption{ \label{Volere5} Requisito no modelo de Volere. }
\end{table}
%%%%%%%%%%%%%%%%%%%%%%%%%%%%%%%%%%%%%%%%%%%%%%%%%%%%%%%%%%%%%%%%%%%%%%%%%%


%%%%%%%%%%%%%%%%%%%%%%%%%%%%%%%%%%%%%%%%%%%%%%%%%%%%%%%%%%%%%%%%%%%%%%%%%%
%%%%%%%%%%%%%%%%%%%%%%%%%%%%% TEMPLATE VOLERE %%%%%%%%%%%%%%%%%%%%%%%%%%%%
\par \noindent
\begin{table}[h]
    \begin{tabularx}{\textwidth}{
        p{\dimexpr.18\linewidth-2\tabcolsep-1.3333\arrayrulewidth}
        p{\dimexpr.1\linewidth-2\tabcolsep-1.3333\arrayrulewidth}
        p{\dimexpr.24\linewidth-2\tabcolsep-1.3333\arrayrulewidth}
        p{\dimexpr.14\linewidth-2\tabcolsep-1.3333\arrayrulewidth}
        p{\dimexpr.255\linewidth-2\tabcolsep-1.3333\arrayrulewidth}
        p{\dimexpr.1\linewidth-2\tabcolsep-1.3333\arrayrulewidth}
    }
    \toprule
        \textbf{Requisito \#:}          & 6
        & \textbf{Tipo de requisito:}   & 2
        & \textbf{Evento/PUC/BUC:}      & 3 \\
    \midrule
    \end{tabularx}
    \vspace{5pt}
    \begin{tabularx}{\textwidth}{p{.12\textwidth}  p{.83\textwidth}}
        \textbf{Descrição:} &
        Visualização do histórico de buscas.
    \end{tabularx}
    
    \vspace{5pt}
    \begin{tabularx}{\textwidth}{p{.12\textwidth}  p{.83\textwidth}}
        \textbf{Razão:} &
        Poder rever, resgatar, reavaliar e/ou lembrar o que foi 
        buscado/visualizado com Dr. Pecúlio.
    \end{tabularx}
    
    \vspace{5pt}
    \begin{tabularx}{\textwidth}{p{.12\textwidth} p{.83\textwidth}}
        \textbf{Origem:} &
        Paulo.
    \end{tabularx}
    
    \vspace{5pt}
    \begin{tabularx}{\textwidth}{p{.23\textwidth} p{.72\textwidth}}
        \textbf{Critério de ajuste:} &
        Uma maneira eficiente de guardar o histórico, ou seja, 
        integração com algum tipo de conta na \emph{internet}, o qual
        permita que o usuário possa acessar de outro computador
        e compartilhar em redes sociais.
    \end{tabularx}
    
    \vspace{5pt}
    \begin{tabularx}{\textwidth}{p{.26\textwidth}p{.248\textwidth}
                                 p{.28\textwidth}p{.110\textwidth}}
        \textbf{Satisfação do cliente:} & -- & 
        \textbf{Insatisfação do cliente:} & --
    \end{tabularx}
    
    \vspace{5pt}
    \begin{tabularx}{\textwidth}{p{.17\textwidth}p{.5\textwidth}
                                 p{.12\textwidth}p{.101\textwidth}}
        \textbf{Dependências:} & Conta de usuário e armazenamento 
        do histórico com dados criptografados para a segurança.
        & \textbf{Conflitos:} & -- \\
    \end{tabularx}
    
    \vspace{5pt}
    \begin{tabularx}{\textwidth}{p{.235\textwidth} p{.715\textwidth}}
        \textbf{Materiais de apoio:} &
       --
    \end{tabularx}
    
    \vspace{5pt}
    \begin{tabularx}{\textwidth}{p{.1\textwidth} p{.85\textwidth}}
        \textbf{História:} &
        Criado em 12 de Junho de 2012. \\
    \end{tabularx}
    
    \begin{tabularx}{\textwidth}{c}
        \bottomrule
    \end{tabularx}
    
    \caption{ \label{Volere6} Requisito no modelo de Volere. }
\end{table}
%%%%%%%%%%%%%%%%%%%%%%%%%%%%%%%%%%%%%%%%%%%%%%%%%%%%%%%%%%%%%%%%%%%%%%%%%%

\FloatBarrier

%%%%%%%%%%%%%%%%%%%%%%%%%%%%%%%%%%%%%%%%%%%%%%%%%%%%%%%%%%%%%%%%%%%%%%%%%%
%%%%%%%%%%%%%%%%%%%%%%%%%%%%% TEMPLATE VOLERE %%%%%%%%%%%%%%%%%%%%%%%%%%%%
\par \noindent
\begin{table}[h]
    \begin{tabularx}{\textwidth}{
        p{\dimexpr.18\linewidth-2\tabcolsep-1.3333\arrayrulewidth}
        p{\dimexpr.1\linewidth-2\tabcolsep-1.3333\arrayrulewidth}
        p{\dimexpr.24\linewidth-2\tabcolsep-1.3333\arrayrulewidth}
        p{\dimexpr.14\linewidth-2\tabcolsep-1.3333\arrayrulewidth}
        p{\dimexpr.255\linewidth-2\tabcolsep-1.3333\arrayrulewidth}
        p{\dimexpr.1\linewidth-2\tabcolsep-1.3333\arrayrulewidth}
    }
    \toprule
        \textbf{Requisito \#:}          & 7
        & \textbf{Tipo de requisito:}   & 2
        & \textbf{Evento/PUC/BUC:}      & 1, 2, 3 \\
    \midrule
    \end{tabularx}
    \vspace{5pt}
    \begin{tabularx}{\textwidth}{p{.12\textwidth}  p{.83\textwidth}}
        \textbf{Descrição:} &
        Acesso ao catálogo das lojas próximas, com base nos resultados 
        da API GeoIP.
    \end{tabularx}
    
    \vspace{5pt}
    \begin{tabularx}{\textwidth}{p{.12\textwidth}  p{.83\textwidth}}
        \textbf{Razão:} &
         Para poder informar preços das lojas mais próximas.
    \end{tabularx}
    
    \vspace{5pt}
    \begin{tabularx}{\textwidth}{p{.12\textwidth} p{.83\textwidth}}
        \textbf{Origem:} &
        Paulo.
    \end{tabularx}
    
    \vspace{5pt}
    \begin{tabularx}{\textwidth}{p{.23\textwidth} p{.72\textwidth}}
        \textbf{Critério de ajuste:} &
        A loja deve existir de maneira \emph{online} e ter informações 
        disponíveis.
    \end{tabularx}
    
    \vspace{5pt}
    \begin{tabularx}{\textwidth}{p{.26\textwidth}p{.248\textwidth}
                                 p{.28\textwidth}p{.110\textwidth}}
        \textbf{Satisfação do cliente:} & -- & 
        \textbf{Insatisfação do cliente:} & --
    \end{tabularx}
    
    \vspace{5pt}
    \begin{tabularx}{\textwidth}{p{.17\textwidth}p{.5\textwidth}
                                 p{.12\textwidth}p{.101\textwidth}}
        \textbf{Dependências:} & Conexão com \emph{internet} e
        a loja ser disponibilizada \emph{online}.
        & \textbf{Conflitos:} &--\\
    \end{tabularx}
    
    \vspace{5pt}
    \begin{tabularx}{\textwidth}{p{.235\textwidth} p{.715\textwidth}}
        \textbf{Materiais de apoio:} &
       --
    \end{tabularx}
    
    \vspace{5pt}
    \begin{tabularx}{\textwidth}{p{.1\textwidth} p{.85\textwidth}}
        \textbf{História:} &
        Criado em 12 de Junho de 2012. \\
    \end{tabularx}
    
    \begin{tabularx}{\textwidth}{c}
        \bottomrule
    \end{tabularx}
    
    \caption{ \label{Volere7} Requisito no modelo de Volere. }
\end{table}
%%%%%%%%%%%%%%%%%%%%%%%%%%%%%%%%%%%%%%%%%%%%%%%%%%%%%%%%%%%%%%%%%%%%%%%%%%

