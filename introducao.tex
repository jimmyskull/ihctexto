
% Em algum momento da introduçào me baseei nisso
% http://people.csail.mit.edu/pcm/tempISWC/workshops/SERES2010/seres10_submission_2.pdf

Somos a sociedade da informação, termo utilizado por
\citeonline{beniger1986control} para referir-se à 
sociedade moderna e seu gradual desejo de controle e consumo da 
informação, desejo este que motivou o
desenvolvimento rápido de tecnologias para o \textit{feedback} automático
em processos industriais, logo no início do período industrial.

Em discussões sobre a sociedade moderna baseadas na dialética do antigo e do novo,
do continuado e do descontinuado, termos como 
capitalismo digital \cite{citeulike:8018090} e
capitalismo virtual \cite{VirtualCapitalism} mostram claramente como
os sistemas computacionais tornaram-se o foco no desenvolvimento tecnológico,
refletindo na criação de uma nova rede cultural e social generalizada, com 
alcance mundial, impactando em uma economia capitalista globalizada.

\citeonline{DBLP:journals/corr/abs-1111-6849} indicam que a geração
de informação cresce exponencialmente.  Como resultado, as pessoas
se veem sobrecarregadas com o volume de informações publicadas em uma
pletora de novos livros e páginas na \textit{internet}, entre outros.  Além disso,
avanços tecnológicos reduziram formidavelmente as barreiras na publicação
e distribuição de informação, através de meios eletrônicos, aumentando a 
dimensão da quantidade de publicações produzidas \cite{rao2008application}.

Na tentativa de amenizar problema do sobrecarregamento de informações e
tentar converter todas as fontes de informação disponíveis em dados úteis,
surgiu o assistente pessoal inteligente (IPA, do inglês \textit{Intelligent
Personal Assistant}) como uma solução factível para auxiliar os usuários
em diferentes domínios de aplicações.

\subsection{Assistente pessoal inteligente}

%{\color{red}
%• Uma descrição sucinta sobre o que você entende por Assistente Pessoal Inteligente 
%para Dispositivos Móveis de forma a definir o escopo de seu projeto. }

Um assistente pessoal inteligente é um agente de \textit{software}
que pode realizar tarefas e serviços para um indivíduo, baseado na
entrada do usuário, como a localização atual e a habilidade para
acessar informações de uma variedade de informações na \textit{internet}, como
condições atuais do clima, do tráfego, notícias, ações, agenda do
usuário e preços de varejo, entre outros \cite{kaschek2007intelligent}.
Exemplos de tais sistemas incluem Google Now \cite{GoogleNow} e 
Siri \cite{Siri}.   Segundo \citeonline{garridoadding}, estes agentes devem 
ser capazes de comunicar, cooperar, discutir com pessoas e também 
guiá-las.

A tecnologia por trás dos assistentes pessoais inteligentes envolve a integração
de dispositivos móveis com \textit{Application Programming Interfaces} (APIs)
públicas,
popularizadas com a proliferação de aplicativos móveis.  
Assistentes inteligentes autônomos são projetados para realizar tarefas
específicas de modo simples através de instruções dadas pelo usuário,
como comandos de voz, redirecionando ações à outros agentes inteligentes,
especializados em realizar tarefas específicas.

Nosso trabalho é voltado ao desenvolvimento da prévia de um agente 
inteligente, especializado em consultar preços de objetos em lojas virtuais
e próximas da localização atual do usuário.

\subsection{Doutor Pecúlio}

Doutor Pecúlio é o nome dado ao aplicativo resultante deste trabalho, o qual
tem o objetivo de agilizar o processo do usuário em consultar preços.
Para interagir, o usuário deve tirar uma foto de um item, que pode ser
por exemplo, a capa de um livro, o código de barras ou o seu código
QR (do inglês, \textit{Quick Response code}).  Em todos os casos, 
necessita-se que o aplicativo
tenha acesso à uma câmera.   O aplicativo é voltado para funcionar com
livros, CDs, DVDs, Blu-ray, mídias para PCs e mídas para consoles de 
\textit{videogame}.

Para uma consulta, precisa-se apenas tirar uma foto da capa do objeto com o 
\textit{smartphone}, ativando a função de câmera.  O usuário pode acessar a
câmera externamente ao programa ou através do menu principal do Doutor Pecúlio.
Com a câmera ativa, é necessário que o usuário alinhe a câmera com o objeto
em uma perspectiva aproximadamente perpendicular.  Após tirar a foto, o 
usuário verifica se a foto ficou boa o suficiente para identificar qual é o
objeto e então inicia a busca.   Para iniciar a busca, existem duas situações.
Se o usuário iniciou a câmera fora do aplicativo, ele deve escolher a opção
de compartilhar a imagem com o Doutor Pecúlio. Porém, se o usuário iniciou
a câmera através do menu do aplicativo, uma opção de confirmar, ou utilizar, a
foto deve ser selecionada.  O aplicativo irá procurar o objeto na 
\textit{internet}, através de uma conexão \textit{Wi-Fi}, onde a consulta
deve ter uma duração aproximada de dois a três segundos antes que os 
resultados comecem a ser mostrados.

Uma vez que uma consulta é realizada com sucesso, o usuário terá acesso
a comentários de avaliações feitos por outros usuários, em \textit{sites} como 
Amazon e Submarino.   Os resultados são ordenados em ordem crescente, sendo
o critério principal o preço do produto e, o secundário, a distância do local, caso o 
resultado seja de uma loja próxima.    Para resultados de lojas locais, 
também são apresentadas direções para chegar e números de telefone, sempre
que possível.

\subsection{Motivação}

Atualmente
existem vários aplicativos móveis que possibilitam a busca de um 
item para consulta e comparação com outras lojas.  Entretanto, muitos necessitam
da entrada manual do usuário, como por exemplo para livros, pode ser
necessário a entrada do identificador ISBN, o autor, o título e
frequentemente detalhes como a
edição do livro que deseja-se consultar.

Com o grande esforço presente na área de pesquisa de visão computacional
relacionados à Recuperação de Imagens Baseada em Conteúdo (CBIR,
do inglês \textit{Content-Based Image Retrieval}) e
Recuperação e Indexação de Imagens Baseada em Conceitos Implícitos (ICIIR,
do inglês, \textit{Implicit Concept-Based Image Indexing and Retrieval}), 
já é possível realizar buscas rápidas por imagens semelhantes em bancos de 
dados grandes, como discutido em \citeonline{jsrgensen2003image},
\citeonline{1265007}, \citeonline{Datta:2008:IRI:1348246.1348248}
e \citeonline{Rui199939}.
\citeonline{Petrakis1993504} discutem os requisitos de \textit{design} e
implementação de um banco de dados de imagens de um sistema que suporta
o armazenamento e recuperação de imagens pelo conteúdo presente nas imagens.

Através destes estudos, acreditamos que seja possível a elaboração de um 
agente inteligente rápido e eficiente, que atenda o usuário de forma 
natural, adequado aos conceitos apresentados por \citeonline{kaschek2007intelligent}, 
sem a necessidade de entrada
manual da descrição do objeto para realizar consultas, almejando ser confortável
e satisfatório ao usuário.

\subsection{Público-alvo}

O usuário que deverá se sentir satisfeito com a usabilidade do aplicativo
é aquele que possui um \textit{smartphone} moderno, com integração à câmera
e conexão \textit{Wi-Fi}.   Imagina-se que este usuário utilizará o
aplicativo em situações corriqueiras, onde o agente causará pouco impacto
na necessidade de atenção ao usuário para sua operação.

% %%
%
%  Atenção: desperdiçando uma referencia passada pela professora:
%   \cite{Rocha}
% %%
  
