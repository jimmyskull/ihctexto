De acordo com a perspectiva de mercado aonde as pessoas cada dia mais tem acesso a 
\emph{smartphones} e a maioria dos \emph{shoppings} tem acesso à rede 
WiFi, além das tecnologias de redes 
móveis e \emph{ad-hoc}, tornam o uso de um aplicativo digital viável, com o crescimento emergente 
do uso de lojas de comercio virtuais por parte de lojas reais, a possibilidade de uso de GeoIP e 
outros métodos de localização, o Dr. Peculio se mostra um sistema viável, com potencial e com 
diferencial no mercado.

O objetivo deste trabalho foi além de mostrar a documentação básica oriunda da 
Engenheira de Software e os requisitos de avaliações por parte da interação 
humano-computador, para que em trabalhos futuros, sirva de referencia para a composição de um 
\emph{software} similar que tome por base esta poderosa ideia.
