
%{\color{red}
%• Uma lista de cenários para os quais se imagina que a aplicação poderá ser aplicada, 
%detalhando 2 cenários principais. }

Um cenário é uma narrativa, a qual descreve todas as interações do usuário
com o sistema durante uma atividade.
Neste trabalho, o objetivo dos cenários é auxiliar a análise de
requisitos funcionais e operacionais do sistema.  Cada cenário descreve como um
sistema deve ser utilizado em um contexto de uma atividade em um prazo de 
tempo definido, sendo que as sub-seções \ref{cenario1} e \ref{cenario2}
descrevem cenários mais detalhados, enquanto que a sub-seção 
\ref{cenario3} descreve um
cenário razoavelmente a alto nível, focando mais em ações feitas ao invés
de operações realizadas a partir de cada ação.

\subsection{Cenário 1} \label{cenario1}

Cenário: Uma tentativa bem sucedida de busca de um produto a partir da
foto da câmera interna do \emph{smartphone}, a qual é utilizada para
tirar uma foto da capa de um livro de interesse do usuário.  O dispositivo
já está conectado à uma rede WiFi.

\begin{enumerate}
\item O usuário inicia a função de câmera;
\item A capa do livro é posicionada de forma aproximadamente ortogonal;
\item O usuário tira uma foto da capa do livro;
\item A foto recém realizada é mostrada ao usuário;
\item O usuário seleciona a opção ``compartilhar'';
\item A opção ``Doutor Pecúlio'' aparece nas opções de compartilhamento,
    a qual é escolhida pelo usuário;
\item O aplicativo ``Doutor Pecúlio'' é iniciado em uma tela onde a foto
    é mostrada escurecida em segundo plano e em primeiro plano é mostrada uma
    mensagem informando que a busca está em andamento;
\item Internamente, como já há conexão estabelecida com a \emph{internet}, o
    algoritmo responsável pelo processamento da imagem é iniciado para identificar
    se a foto é uma capa ou um código de barras ou QR.  O algoritmo identifica que 
    é uma capa e então um segundo algoritmo é executado para criar uma
    identificação única (\emph{hash}) da imagem.   O identificador único é
    utilizado para realizar uma consulta em um servidor dedicado ao aplicativo,
    o qual irá retornar possíveis resultados.  Neste caso, apenas um resultado
    foi encontrado os resultados para aquele título é retornado ao aplicativo.
\item Em aproximadamente três segundos os resultados do livro começam 
    aparecer em uma tela onde mostra a capa \emph{online} do livro e
    resultados encontrados.  Ao lado da capa, o título encontrado é apresentado,
    e logo abaixo a avaliação dos usuários é pela quantidade de estrelas.
    Os resultados são divididos em lojas virtuais
    e lojas locais.  As lojas locais são organizadas em ordem crescente de
    distância e preço, mostradas na tela do usuário.   O usuário pode acessar
    todos os resultados \emph{online} a partir de uma opção mostrada depois
    do título e antes dos resultados locais.    
\end{enumerate}

\subsection{Cenário 2} \label{cenario2}


Cenário: Uma tentativa mal sucedida de busca de um produto a partir da
foto da câmera interna do \emph{smartphone}, a qual é utilizada para
tirar uma foto do código de barras de um DVD de interesse do usuário.  O dispositivo
já está conectado à uma rede WiFi.

\begin{enumerate}
\item O usuário inicia a função de câmera;
\item O código de barras é posicionado de forma aproximadamente ortogonal;
\item O usuário tira uma foto;
\item A foto recém realizada é mostrada ao usuário;
\item O usuário seleciona a opção ``compartilhar'';
\item A opção ``Doutor Pecúlio'' aparece nas opções de compartilhamento,
    a qual é escolhida pelo usuário;
\item O aplicativo ``Doutor Pecúlio'' é iniciado em uma tela onde a foto
    é mostrada escurecida em segundo plano e em primeiro plano é mostrada uma
    mensagem informando que a busca está em andamento;
\item Internamente, como já há conexão estabelecida com a \emph{internet}, o
    algoritmo responsável pelo processamento da imagem é iniciado para identificar
    se a foto é uma capa ou um código de barras ou QR.  O algoritmo identifica que 
    é um código de barras e então um segundo algoritmo é executado para ler o
    código de barras da imagem.   O código de barras é
    utilizado para realizar uma consulta em um servidor dedicado ao aplicativo,
    o qual irá retornar os resultados.  Porém, este DVD ainda não está disponível
    pelas APIs.  O servidor retorna um erro; e
\item Em aproximadamente três segundos a mensagem de erro é mostrada com a
    foto em segundo plano e em primeiro plano é mostrada uma mensagem mostrando
    que o item não foi encontrado.  O usuário tem a opção de poder realizar 
    uma novo foto ou cancelar a operação de busca.
\end{enumerate}

\subsection{Cenário 3} \label{cenario3}

Cenário: Acesso à uma busca já realizada anteriormente.  O dispositivo
não precisa estar conectado à WiFi.

\begin{enumerate}
\item O usuário inicia o aplicativo ``Doutor Pecúlio'';
\item A tela de consultas já realizada é mostrada ao usuário.  Os
    resultados bem sucedidos são mostrados de forma agrupada por tipo 
    de produto consultado, como Livro, DVD e Jogos;
\item O usuário seleciona um resultado de uma busca anterior; e
\item Os mesmos resultados utilizados na busca anterior são mostrados,
    e uma opção de atualizar os resultados é mostrada para que ele possa
    buscar por novos preços e disponibilidade na localidade atual do
    usuário.
\end{enumerate}
